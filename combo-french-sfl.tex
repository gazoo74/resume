%% combo-french-sfl.tex
%% Copyright 2015 Gaël PORTAY <gael.portay@gmail.com>
%
% This work may be distributed and/or modified under the
% conditions of the LaTeX Project Public License, either version 1.3
% of this license or (at your option) any later version.
% The latest version of this license is in
%   http://www.latex-project.org/lppl.txt
% and version 1.3 or later is part of all distributions of LaTeX
% version 2005/12/01 or later.
%
% This work has the LPPL maintenance status `maintained'.
%
% The Current Maintainer of this work is Gaël PORTAY.
%
% This work consists of the files combo-french-sfl.tex and gael.eps.

\documentclass[11pt,a4paper,sans]{moderncv}

\moderncvstyle{casual}
\moderncvcolor{blue}

\usepackage[scale=0.75]{geometry}
\usepackage[utf8]{inputenc}
\usepackage[francais]{babel}

\name{Gaël}{PORTAY}
\title{Ingénieur Linux Embarqué}
\address{12 chemin de Mireille}{74200 Thonon-les-bains}{France}
\phone[mobile]{+33 (0)6 63 76 57 11}
\email{gael.portay@gmail.com}
\social[linkedin]{gaël.portay}
\social[twitter]{gazoo74}
\social[github]{gazoo74}
\photo[64pt][0.4pt]{profile.eps}

\recipient{Savoir Faire Linux}{7275 Saint Urbain - Bureau 200\\Montréal (Québec) H2R 2Y5\\Canada}
\date{\today}
\opening{Bonjour,}
\closing{Je vous remercie par avance de l'attention que vous porterez à ma candidature. Je reste à votre entière disposition pour toute information complémentaire ou pour convenir d'un rendez-vous pour un entretien. Dans l’attente de votre réponse, je vous prie d’agréer mes sincères salutations.}
\enclosure[Ci-joint]{Curriculum Vit\ae{}.}

\begin{document}
\makelettertitle

Je vous adresse cette lettre suite à la publication de vos offres d'emploi sur votre \href{https://carrieres.savoirfairelinux.com/}{site Internet}. En effet le poste de « \href{https://carrieres.savoirfairelinux.com/#consultant-logiciel-pour-systemes-embarques}{\textit{Consultant logiciel pour systèmes embarqués}} » a particulièrement retenu mon attention.

Ces cinq dernières années, j'ai acquis de réelles connaissances dans le domaine de \textit{Linux embarqué}. Au niveau espace utilisateur tout d'abord, en développant des applications et des librairies pour des systèmes embarqués (C/C++, epoll, compilation-croisée...). Puis j'ai approfondi mes compétences en système d'exploitation en développant une distribution \textit{Linux} maison depuis zéro (système d'initialisation, gestionnaire de paquets, gestion dynamique des périphériques...). Plus récemment, j'ai travaillé dans le noyau \textit{Linux} et sur les étapes de démarrage d'un tel système lorsque j'ai développé le logiciel bas-niveau (\textit{BSP}) de deux nouvelles cartes électroniques à base de système-sur-puce (\textit{SoC}) ARM.

Ces expériences m'ont confronté plusieurs fois à la communauté du \textit{Logiciel Libre} si bien que je souhaite marquer un tournant dans ma carrière. Aujourd'hui, je cherche à évoluer vers un poste de consutant afin de partager à la fois mes connaissances et mes développements. Vous trouverez de plus amples informations sur mon cursus à travers mon \href{http://portay.fr/journees-quebec/pdf/french.pdf}{\textit{CV}} et sur mes compétences techniques via mon \href{http://portay.fr/journees-quebec/portfolio-french/index.html}{\textit{portfolio}}.

\makeletterclosing

\clearpage

\makecvtitle

\section{Experiences}
\subsection{Professionnelles}
\cventry{Juillet 2010--Septembre 2015}{Ingénieur Linux Embarqué}{Overkiz SAS, groupe Somfy}{Archamps}{}{Overkiz est une entreprise spécialisée dans le « Cloud Computing » pour la domotique. Elle propose une solution qui connecte les objets de la maison à Internet (IoT). Elle se compose d'une passerelle, appelée Kizbox, qui fait le lien entre les périphériques domotiques et ses serveurs. Il est possible de piloter ses objets de la maison grâce à son smart-phone et des Web-Services. J'ai fait parti de l'équipe qui développe le système Linux embarqué par la passerelle.\newline{}
\begin{itemize}
\item J'étais co-responsable de la distribution Linux embarquée. J’intégrai des outils issus de la communauté du logiciel libre.
\item J'étais responsable du déploiement des mises-à-jour du logiciel embarqué.
\item J’ai mis en place le système de « Build automatisé » Yocto. Cet outil construit l’intégralité du système embarqué de la passerelle. Yocto permet de gagner de nombreuses heures à l’équipe tout en minimisant les erreurs liées à des interventions humaines.
\item J'ai développé des framework internes et des applications ajoutant la prise en charge de nouveaux protocoles domotiques. J’ai été en charge du développement de l’application responsable du maintien de la connexion entre la passerelle et le serveur. Les frameworks et les applications sont développés en C++.
\item J'ai développé des modules noyaux et ajouté le support des nouvelles plateformes matérielles développées par notre équipe dans le noyau Linux.
\item J'ai contribué sur des projets Open-Source (voir la section Contributions Open-Source).
\end{itemize}}

\subsection{Stages}
\cventry{2009}{Stage de Master}{LC Mobility}{Australie}{}{LC Mobility est une entreprise spécialisée dans l’accueil de doctorant. J’ai effectué une étude de marché sur la mobilité des étudiants Australiens, en Australie.}
\cventry{2008}{Stage Ingénieur de 3e année}{Freescale Semiconductors}{Toulouse}{}{J'ai été responsable du développement du pilote de charge de batterie de la platforme de téléphonie mobile de Freescale (ARM-11). Cette plateforme est alimentée par une batterie Li-ion. Elle fonctionne sous Nokia S60 (Symbian OS). J'ai developpé le pilote en C++ en utilisant les mécanismes propres à Symbian.}
\cventry{2007}{Stage Ingénieur de 2ème année}{Sagem Monetel}{Valence}{}{J’ai été chargé d’optimiser l’emprunte mémoire (dynamique et statique) d'une application C embarquée dans un terminal de paiement banquaire. J’ai utilisé les fichiers listing et mapping générés par le compilateur libre GNU/GCC pour localiser les parties de code utilisant le plus de ressources mémoires. J’ai réduit la taille de l’application de plus de 30\%.}
\cventry{2004}{Stage Technicien de 2ème année de DUT}{Sagem Monetel}{Valence}{}{J’ai eut la charge de porter une application C embarquée vers le compilateur libre GNU/GCC. J’ai également développé un outil de test de performances afin de démontrer la puissance du terminal de paiement durant une transaction banquaire.}

\subsection{Contributions Open-Source}
\cvitem{\href{https://git.kernel.org/cgit/linux/kernel/git/torvalds/linux.git/log/?qt=grep&q=PORTAY}{Noyau Linux}}{J'ai ajouté deux nouvelles platformes basées sur des SoC d'Atmel (device-tree).}
\cvitem{\href{https://github.com/linux4sam/at91bootstrap/commits?author=gazoo74}{Atmel at91bootstrap}}{J'ai ajouté le support d'UBI. Le but est d'améliorer les mises-à-jour critiques vis-à-vis d'éventuelles coupures de courants. Les volumes critiques, tels que les noyaux ou les bootloaders, sont dupliqués et stockés dans des volumes UBI statiques. Le bootstrap vérifie simplement l'intégrité du volume en utilisant le bit de mise-à-jour présent dans l'en-tête UBI.}
\cvitem{\href{https://github.com/bagder/curl/commits?author=gazoo74}{CURL}}{J'ai mis-à-jour libcurl afin qu'il soit compatible avec la dernière version des API de la librairie PolarSSL. J'ai également corrigé un bogue avec le mecanisme de polling qui entrainait timeout lors de la négociation SSL avec le serveur distant.}
\cvitem{\href{http://git.yoctoproject.org/cgit/cgit.cgi/opkg/log/?qt=grep&q=PORTAY}{OPKG}}{J'ai améliorer l'intégration de CURL en autorisant certaines options liés à libcurl dans le fichier de configuration. J'ai également corrigés des comportements inattendus.}
\cvitem{\href{https://github.com/mkj/dropbear/commits?author=gazoo74}{Dropbear}}{J'ai corrigé les warnings de compilation du projet.}

\subsection{Divers}
\cventry{2000--2006}{Emplois saisonniers}{Plusieurs employeurs}{}{}{}

\section{Education}
\cventry{2008--2009}{Master MAE}{IAE}{Grenoble}{\textit{Bac +5}}{Année spéciale de Master en Management des Administrations et des Entreprises.}
\cventry{2005--2008}{3I}{Polytech'Grenoble}{Grenoble}{\textit{Bac +5}}{Diplôme d’ingénieur en Informatique Industrielle et Instrumentation (France) – 5 ans
e.}
\cventry{2004--2005}{Licence TIC}{Université de Savoie}{Chambéry}{\textit{Bac +3}}{2ème année de Licence en Technologie de l’Information et de la Communication.}
\cventry{2002--2004}{DUT ISI}{IUT de Valence}{Valence}{\textit{Bac +2}}{Diplôme universitaire de Technologie en Informatique et Systèmes Industriels.}
\cventry{2001--2002}{DEUG SV}{Université de Savoie}{Chambéry}{\textit{Bac +2}}{1ère année de Diplôme d’Etudes Universitaire Général en Science de la Vie.}

\section{Projets}
\cventry{2009}{LHOG Minatec}{Amplificateur 900 MHz}{Grenoble}{}{Conception d'un amplificateur GSM 900 au LHOG Minatec (laboratoire de micro-nano technologie). Design, simulation, routage, assemblage, tests and caractérisation.}
\cventry{2008}{LHOG Minatec}{Convertisseur Analogique-Numérique CMOS 6 bits}{Grenoble}{}{Conception d'un convertisseur analogique-numérique 6-bits utilisant la technologie 0,35u au CIME Minatec (micro-nano laboratory). Les comparateurs CMOS ont été conçus via le logiciel Cadense, le correcteur et le décodeur ont été développé en VHDL.}
\cventry{2006}{Polytech'Grenoble}{Assembleur HC12}{Grenoble}{}{Conception d'un assembleur 2-passes pour le jeu d'instruction 68HC12. Cette outil en ligne de commande a été developpé en C sur Linux.}
\cventry{2005}{Université de Savoie}{Windows Desktop Search}{Chambéry}{}{Conception d'un moteur de recherche rapide pour Windows. J'ai developpé cette application en Java sous l'environnement de développement Eclipse. Le moteur indexe tous les fichiers de l'ordinateur et permet à l'utilisateur de rechercher des fichiers via des expressions régulières. L'UI a été developpée via le framework Swing.}
\cventry{2004}{IUT de Valence}{Moteur de jeu 2D}{Valence}{}{Conception d'un moteur de jeu 2D basic en utilisant les librairies Microsoft Direct Draw. Le joueur se déplace sur une carte 2D avec gestion des collisions. J'ai développé le moteur en C++.}

\section{Langues}
\cvitemwithcomment{Français}{Langue maternelle}{}
\cvitemwithcomment{Anglais}{Bonnes connaissances}{J'ai voyagé un an en Australie et Nouvelle-Zélande (d'avril 2009 à mars 2010). J'ai obtenu 765 au TOEIC in 2008.}

\section{Compétences techniques}
\cvitem{Outils informatiques}{Microsoft Office, Eclipse, Xcode, LaTeX}
\cvitem{Languages de programmation}{Scripts Shell/Bash, Python, C/C++, STL, Assembleurs (68k, MIPS), Java, VHDL}
\cvitem{Autres}{Linux Kernel, Git, Autotools, Cross-compilation, Yocto}

\section{Centres d'intérêts}
\cvitem{BDE/BDS Polytech'Grenoble}{Membre du bureau des étudiants de Polytech’Grenoble.\newline Organisation du week-end d'intégration en 2006 (350 étudiants, 3 jours).\newline Organisation de deux week-end ski en 2006 et 2007 (400 étudiants, 3 jours, 11 écoles d'ingénieurs du réseau Polytech).}
\cvitem{Montagne}{Ski et randonnées.}
\cvitem{Tennis de table}{Joueur, arbitre et entraineur (9 ans).}
\cvitem{Photographie amateur}{Canon EOS 500D}

\section{Références}
\cvitem{Overkiz, SAS.}{Florent PELLARIN, Directeur des Opérations (\href{mailto:f.pellarin@overkiz.com}{f.pellarin@overkiz.com})}
\end{document}
