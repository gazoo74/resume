%% french-short.tex
%% Copyright 2015 Gaël PORTAY <gael.portay@gmail.com>
%
% This work may be distributed and/or modified under the
% conditions of the LaTeX Project Public License, either version 1.3
% of this license or (at your option) any later version.
% The latest version of this license is in
%   http://www.latex-project.org/lppl.txt
% and version 1.3 or later is part of all distributions of LaTeX
% version 2005/12/01 or later.
%
% This work has the LPPL maintenance status `maintained'.
%
% The Current Maintainer of this work is Gaël PORTAY.
%
% This work consists of the files french-short.tex and gael.eps.

\documentclass[11pt,a4paper,sans]{moderncv}

\moderncvstyle{classic}
\moderncvcolor{blue}

\usepackage[scale=0.75]{geometry}
\usepackage[utf8]{inputenc}

\name{Gaël}{PORTAY}
\title{Ingénieur Linux Embarqué}
\address{12 chemin de Mireille}{74200 Thonon-les-bains}{France}
\phone[mobile]{+33 (0)6 63 76 57 11}
\email{gael.portay@gmail.com}
\social[linkedin]{gaël.portay}
\social[twitter]{gazoo74}
\social[github]{gazoo74}
\photo[64pt][0.4pt]{profile.eps}

\begin{document}
\makecvtitle

\section{Experiences}
\subsection{Professionnelles}
\cventry{Juillet 2010--Septembre 2015}{Ingénieur Linux Embarqué}{Overkiz SAS, groupe Somfy}{Archamps}{}{
\begin{itemize}
\item Co-responsable de la distribution Linux embarqué
\item Intégration d'outils issus de la communauté du logiciel libre
\item Responsable des mises-à-jour du logiciel embarqué (OPKG, paquets Debian)
\item Mise en place du système de « Build automatisé » Yocto (Python, shell)
\item Développement de frameworks et d'applications (C/C++)
\item Développement de modules noyaux Linux (C)
\item Ajout de nouvelles plateformes matérielles dans le noyau Linux (device-tree, Git)
\item Contributions sur des projets Open-Source (cf. Contributions Open-Source)
\end{itemize}}

\subsection{Stages}
\cventry{2009}{Stage de Master}{LC Mobility}{Australie}{}{Etude de marché sur la mobilité des étudiants Australiens, en Australie}
\cventry{2008}{Stage Ingénieur de 3e année}{Freescale Semiconductors}{Toulouse}{}{Développement d'un pilote de charge de batterie Lithium-ion sous Nokia S60 (Symbian OS)}
\cventry{2007}{Stage Ingénieur de 2ème année}{Sagem Monetel}{Valence}{}{Optimisation de l’emprunte mémoire d'une application C embarquée ; 30\% de gain}
\cventry{2004}{Stage Technicien de 2ème année de DUT}{Sagem Monetel}{Valence}{}{
\begin{itemize}
\item Portage d'application C embarquée vers le compilateur libre GNU/GCC
\item Développement d'un outil de benchmarking
\end{itemize}}

\subsection{Contributions Open-Source}
\cvitem{\href{https://git.kernel.org/cgit/linux/kernel/git/torvalds/linux.git/log/?qt=grep&q=PORTAY}{Linux}}{Ajout de deux nouvelles platformes (device-tree, SoC Atmel)}
\cvitem{\href{https://github.com/linux4sam/at91bootstrap/commits?author=gazoo74}{AT91Bootstrap}}{Ajout du support UBI pour securisation des mises-à-jour de volumes (par duplication)}
\cvitem{\href{https://github.com/bagder/curl/commits?author=gazoo74}{CURL}}{
\begin{itemize}
\item Ajout de la compatibilité des nouvelles API de la librairie PolarSSL
\item Correction d'un bogue avec le mécanisme de polling pour la librairie PolarSSL
\end{itemize}}
\cvitem{\href{http://git.yoctoproject.org/cgit/cgit.cgi/opkg/log/?qt=grep&q=PORTAY}{OPKG}}{
\begin{itemize}
\item Amélioration de l'intégration d'options libcurl dans le fichier de configuration
\item Correction de comportements inattendus
\end{itemize}}
\cvitem{\href{https://github.com/mkj/dropbear/commits?author=gazoo74}{Dropbear}}{Correction de tous les warnings de compilation}

\subsection{Divers}
\cventry{2000--2006}{Emplois saisonniers}{Plusieurs employeurs}{}{}{}

\section{Education}
\cventry{2008--2009}{Master MAE}{IAE}{Grenoble}{\textit{Bac +5}}{Année spéciale de Master en Management des Administrations et des Entreprises}
\cventry{2005--2008}{3I}{Polytech'Grenoble}{Grenoble}{\textit{Bac +5}}{Diplôme d’ingénieur en Informatique Industrielle et Instrumentation}
\cventry{2004--2005}{Licence TIC}{Université de Savoie}{Chambéry}{\textit{Bac +3}}{2ème année de Licence en Technologie de l’Information et de la Communication}
\cventry{2002--2004}{DUT ISI}{IUT de Valence}{Valence}{\textit{Bac +2}}{Diplôme universitaire de Technologie en Informatique et Systèmes Industriels}
\cventry{2001--2002}{DEUG SV}{Université de Savoie}{Chambéry}{\textit{Bac +2}}{1ère année de Diplôme d’Etudes Universitaire Général en Science de la Vie}

\section{Projets}
\cventry{2009}{LHOG Minatec}{Amplificateur 900 MHz}{Grenoble}{}{Conception d'un amplificateur GSM 900 au LHOG Minatec : design, simulation, routage, assemblage, tests and caractérisation}
\cventry{2008}{LHOG Minatec}{Convertisseur Analogique-Numérique CMOS 6 bits}{Grenoble}{}{Conception d'un convertisseur analogique-numérique 6-bits utilisant la technologie 0,35u au CIME Minatec : comparateurs CMOS (Cadense), correcteur/décodeur (VHDL)}
\cventry{2006}{Polytech'Grenoble}{Assembleur HC12}{Grenoble}{}{Conception d'un assembleur 2-passes pour 68HC12 (C, Linux)}
\cventry{2005}{Université de Savoie}{Windows Desktop Search}{Chambéry}{}{Conception d'un moteur de recherche rapide pour Microsoft Windows XP par indexation (Java, Swing, Eclipse, Regex)}
\cventry{2004}{IUT de Valence}{Moteur de jeu 2D}{Valence}{}{Conception d'un moteur de jeu 2D basic avec gestion de collisions (C++, Microsoft Direct Draw)}

\section{Langues}
\cvitemwithcomment{Français}{Langue maternelle}{}
\cvitemwithcomment{Anglais}{Bonnes connaissances}{Un an de voyage en Australie et Nouvelle-Zélande (avril 2009 à mars 2010) ; 765 au TOEIC (2008)}

\section{Compétences techniques}
\cvitem{Outils}{Microsoft Office, Eclipse, Xcode, LaTeX}
\cvitem{Languages}{Scripts Shell/Bash, Python, C/C++, STL, Assembleurs (68k, MIPS), Java, VHDL}
\cvitem{Autres}{Linux Kernel, Git, Autotools, Cross-compilation, Yocto}

\section{Centres d'intérêts}
\cvitem{Polytech'Grenoble}{Membre du bureau des étudiants et bureau des sports (BDE/BDS)
\begin{itemize}
\item Organisation du week-end d'intégration en 2006 (350 étudiants, 3 jours)
\item Organisation de deux week-end ski en 2006 et 2007 (400 étudiants, 3 jours, 11 écoles d'ingénieurs du réseau Polytech)
\end{itemize}}
\cvitem{Sport}{Ski, randonnées ; Tennis de table (joueur, arbitre, entraineur, 9 ans)}
\cvitem{Photographie amateur}{Canon EOS 500D}

\section{Références}
\cvitem{Overkiz, SAS.}{Florent PELLARIN, Directeur des Opérations (\href{mailto:f.pellarin@overkiz.com}{f.pellarin@overkiz.com})}
\end{document}
