%% portfolio-french.tex
%% Copyright 2015 Gaël PORTAY <gael.portay@gmail.com>
%
% This work may be distributed and/or modified under the
% conditions of the LaTeX Project Public License, either version 1.3
% of this license or (at your option) any later version.
% The latest version of this license is in
%   http://www.latex-project.org/lppl.txt
% and version 1.3 or later is part of all distributions of LaTeX
% version 2005/12/01 or later.
%
% This work has the LPPL maintenance status `maintained'.
%
% The Current Maintainer of this work is Gaël PORTAY.
%
% This work consists of the file portfolio-french.tex.

\documentclass[a4paper]{article}
\usepackage{hyperref}
\usepackage[T1]{fontenc}
\usepackage[utf8]{inputenc}

\title{Portfolio}
\author{Gaël PORTAY}
\date{\today}

\begin{document}
\maketitle

\begin{abstract}
Je suis ingénieur en informatique, spécialisé en \textit{Linux Embarqué}. J'ai découvert l'informatique à l'âge de 10 ans. Aujourd'hui je totalise un peu plus de 6 ans d'experience dans le monde professionnel. Je suis passionné et curieux : j'aime apprendre et comprendre comment fonctionne les «~choses~». Je suis également auto-didacte et j'aime partager mes experiences et mes connaissances avec les autres.\\

Mon travail étant la partie immergée de l'iceberg, il est par conséquent assez difficile de montrer le fruit de mon travail par des images ou des photos. En lieu et place\footnote{Plus tard, j'illustrerai mes propos par quelques schémas.}, j'expliquerai comment et avec quels outils j'ai réalisé mes travaux. De même, certains de mes développements étant du domaine du propriétaire, je ne présenterai ici que mes développements libres.\\

Je mets en évidence deux projets pour démontrer mes compétences : le \textit{BSP}\footnote{Board Support Package.} pour la partie système d'exploitation au niveau noyau et la partie électronique ; les projets personnels pour la partie système d'exploitation au niveau espace utilisateur. L'annexe montre mon implication vis-à-vis de la communauté du Logiciel Libre.\\

Voici une liste non exhaustive de mes compétences : langage \textit{C/C++} (libc, STL), mécanisme de \textit{polling} (epoll), \textit{Git}, \textit{Systèmes Linux} (espace utilisateur et noyau), scripts \textit{Shell} (POSIX, redirection, pipe...), \textit{Python}, \textit{Makefile}, \textit{Autotools}, \textit{Kconfig}, \textit{Yocto}, \textit{crosstool-ng}, \textit{QEMU}...
\end{abstract}
\clearpage

\tableofcontents
\clearpage

\appendix
\section{Contributions}
Voici une liste de mes contributions dans différents projets libres :
\begin{itemize}
\item \textbf{Linux} : \url{https://git.kernel.org/cgit/linux/kernel/git/next/linux-next.git/log/?id=refs\%2Ftags\%2Fnext-20150610&qt=grep&q=PORTAY}
\item \textbf{OPKG} : \url{http://git.yoctoproject.org/cgit/cgit.cgi/opkg/log/?qt=grep&q=PORTAY}
\item \textbf{cURL} : \url{https://github.com/bagder/curl/commits?author=gazoo74}
\item \textbf{AT91Bootstrap} :
\begin{itemize}
\item \url{https://github.com/linux4sam/at91bootstrap/commits?author=gazoo74}
\item \url{https://github.com/linux4sam/at91bootstrap/pull/25}
\end{itemize}
\item \textbf{Dropbear} : \url{https://github.com/mkj/dropbear/commits?author=gazoo74}
\end{itemize}
Ainsi que lien vers mes dépots hebergés par GitHub : \url{https://www.github.com/gazoo74}.
\end{document}
