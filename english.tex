%% english.tex
%% Copyright 2015 Gaël PORTAY <gael.portay@gmail.com>
%
% This work may be distributed and/or modified under the
% conditions of the LaTeX Project Public License, either version 1.3
% of this license or (at your option) any later version.
% The latest version of this license is in
%   http://www.latex-project.org/lppl.txt
% and version 1.3 or later is part of all distributions of LaTeX
% version 2005/12/01 or later.
%
% This work has the LPPL maintenance status `maintained'.
%
% The Current Maintainer of this work is Gaël PORTAY.
%
% This work consists of the files english.tex and gael.eps.

\documentclass[11pt,a4paper,sans]{moderncv}

\moderncvstyle{casual}
\moderncvcolor{blue}

\usepackage[scale=0.75]{geometry}
\usepackage[utf8]{inputenc}

\name{Gaël}{PORTAY}
\title{Embedded Linux Engineer}
\address{12 chemin de Mireille}{74200 Thonon-les-bains}{France}
\phone[mobile]{+33 (0)6 63 76 57 11}
\email{gael.portay@gmail.com}
\social[linkedin]{gaël.portay}
\social[twitter]{gazoo74}
\social[github]{gazoo74}
\photo[64pt][0.4pt]{profile.eps}

\begin{document}
\makecvtitle

\section{Working experiences}
\subsection{Vocational}
\cventry{July 2010--September 2015}{Embedded Linux Engineer}{Overkiz SAS, Somfy group}{Archamps}{}{Overkiz is developing home-automation solutions to control different home devices over Internet. It consists of a gateway and server applications. The gateway runs an Embedded Linux system.\newline{}
\begin{itemize}
\item I am co-maintainer of our homemade Embedded Linux distribution. I am responsible of the software deployment for the embedded part.
\item I set up the Yocto Build System that builds the software of the box. It builds from scratch the whole embedded software.
\item I develop homemade frameworks and applications to support new RF protocols into our gateways. I was also in charge of developing the application that creates the connection between the box and the server. Frameworks and applications are both developed in C++.
\item I have developed Linux kernel modules and I did board bring up and maintain them inside the kernel sources.
\item I have also contributed to Open-Source projects (see Open-Source Contributions section).
\end{itemize}}

\subsection{Training Periods}
\cventry{2009}{Trainee}{LC Mobility}{Australia}{}{I made a market search about the mobility of the Australian students.}
\cventry{2008}{Trainee Engineer}{Freescale Semiconductors}{Toulouse}{}{I was responsible for the development of a battery-charging driver on Freescale smart phone platform (ARM-11 based). This platform is power supplied by a Li-ion battery and uses the Freescale MXC13783 IC as Power Management IC. It is running on Nokia S60 (Symbian OS), and I developed the driver in C++ using Symbian mechanisms.}
\cventry{2007}{Trainee Engineer}{Sagem Monetel}{Valence}{}{I was in charge of memory optimization (dynamic and static) of an embedded application. This payment application is programmed into a banking terminal powered by two ARM processors. I used the listing and map files generated by GNU/GCC compiler to locate heavy memory structures. I reduced the application size by 30\%.}
\cventry{2004}{Technical Trainee}{Sagem Monetel}{Valence}{}{I was in charge of changing the compiler of an embedded application to the GNU/GCC compiler. I also developed a benchmark to demonstrate the power/fast of a payment terminal during a transaction.}

\subsection{Open-Source Contributions}
\cvitem{\href{https://git.kernel.org/cgit/linux/kernel/git/next/linux-next.git/log/?id=refs\%2Ftags\%2Fnext-20150610&qt=grep&q=PORTAY}{Linux Kernel}}{I added two Atmel SoC based device-trees.}
\cvitem{\href{https://github.com/linux4sam/at91bootstrap/commits?author=gazoo74}{Atmel at91bootstrap}}{I brought support for UBI. The goal is to improve critical upgrades against unexpected powercuts. Critical volumes, such as kernels or bootloaders, are duplicated and stored in UBI static volumes. The bootstrap simply checks the volume integrity using update flag from UBI headers.}
\cvitem{\href{https://github.com/bagder/curl/commits?author=gazoo74}{CURL}}{I upgraded libcurl to use the latest PolarSSL Library API. I also fixed a bug with the polling mecanism that causes a timeout while processing SSL handshake with distant server.}
\cvitem{\href{http://git.yoctoproject.org/cgit/cgit.cgi/opkg/log/?qt=grep&q=PORTAY}{OPKG}}{I improved CURL integration by allowing to set libcurl related options into configuration file. I also fixed unexpected behaviors.}
\cvitem{\href{https://github.com/mkj/dropbear/commits?author=gazoo74}{Dropbear}}{I removed warnings from the entire project.}

\subsection{Miscellaneous}
\cventry{2000--2006}{Casual Jobs}{Miscellaneous Employers}{}{}{}

\section{Education}
\cventry{2008--2009}{Master MAE}{IAE}{Grenoble}{\textit{5-year University degree}}{Last year in Master in Management and Business of Administration.}
\cventry{2005--2008}{Master 3I}{Polytech'Grenoble}{Grenoble}{\textit{5-year University degree}}{Engineering degree in Industrial Computing and Microelectronic.}
\cventry{2004--2005}{DEUG TIC}{University of Savoy}{Chambéry}{\textit{2-year University degree}}{in IT.}
\cventry{2002--2004}{DUT ISI}{IUT}{Valence}{\textit{2-year Academic and Technical degree}}{in Industrial Computing.}
\cventry{2001--2002}{DEUG SV}{University of Savoy}{Chambéry}{\textit{2-year University}}{First year of a 2-year University degree in Biology.}

\section{School projects}
\cventry{2009}{LHOG Minatec}{900 MHz amplifier}{Grenoble}{}{Conception of a GSM 900 amplifier at LHOG Minatec (micro-nano technology laboratory). Design, simulation, layout, assembly, tests and characterization.}
\cventry{2008}{LHOG Minatec}{CMOS 6 bits accuracy ADC}{Grenoble}{}{Conception of an Analog to Digital Converter in AMS CMOS 0,35u technology at CIME Minatec (micro-nano  technology laboratory). CMOS comparators had been designed using Cadense, and the Corrector and the Decoder developped in VHDL.}
\cventry{2006}{Polytech'Grenoble}{ASM HC12 assembler}{Grenoble}{}{Conception of an assembler for the 68HC12 instruction set. This command line tool was developed in C on Linux.}
\cventry{2005}{University of Savoy}{Windows Desktop Search}{Chambéry}{}{Conception of Desktop Search for Windows. I developed that application in Java using the Eclipse environment. It indexes all files from the computer and enables searching files by file name using regular expressions (Regex). The UI was build thanks Swing framework.}
\cventry{2004}{IUT of Valence}{2D Game Engine}{Valence}{}{Conception of a simple 2D game engine using Microsoft Direct Draw library. The player moves in a 2D map with collision handling. I developed the engine in C++.}

\section{Languages}
\cvitemwithcomment{French}{Mother tongue}{}
\cvitemwithcomment{English}{Good knowledge}{I spent one year travelling around Australia and New-Zealand (from April 2009 to March 2010). I scored 765 at TOEIC in 2008.}

\section{Computer skills}
\cvitem{Computer literacy}{Microsoft Office, Eclipse, Xcode, Latex}
\cvitem{Programming languages}{Shell/Bash scripting, Python, C/C++, STL, Assemblers (68k, MIPS), Java, VHDL}
\cvitem{Others}{Linux Kernel, Git, Autotools, Cross-compilation, Yocto}

\section{Interests}
\cvitem{BDE/BDS Polytech'Grenoble}{Member of Student Association at Polytech’Grenoble.\newline Organization of the Integration event in 2006 (350 students, 3 days).\newline Organization of two Ski event in 2006 and 2007 (400 students, 3 days, 11 engineering school from France).}
\cvitem{Mountain}{Ski and hiking.}
\cvitem{Table tennis}{Player, referee and coach (9-year practicing).}
\cvitem{Photography}{Landscape, Canon EOS 500D}

\section{References}
\cvitem{Overkiz, SAS.}{Florent PELLARIN, Chief Operational Officer (\href{mailto:f.pellarin@overkiz.com}{f.pellarin@overkiz.com})}
\end{document}
